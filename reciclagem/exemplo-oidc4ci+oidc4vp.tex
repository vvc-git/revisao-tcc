Nesse exemplo [\ref{fig:OID4VC}], agrupados as duas modificações para implementação da extensão do OpenID para Credenciais Verificáveis, o \acs{OIDC4VP} e \acs{OIDC4CI}. Para isso, foi utilizado o fluxo de entre dispositivos (cross device flow) para o sub-protocolo \acs{OIDC4VP} e o fluxo de código de Autorização (authorization code flow) para o sub-protocolo \acs{OIDC4CI}. No fluxo entre dispositivos, O verificador prepara uma Solicitação de Autorização e a renderiza como um QR Code. O usuário então usa a carteira para escanear o Código QR Code. É importante observar também que nesse exemplo específico a emissão de credenciais foi feita sob-demanda, ou seja, somente quando a carteira digital tem a \acs{DA}, que ela requisita um credencial para o Provedor de Identidade. 

\begin{figure}[htb]
  \caption{Diagrama de sequência para o protocolo OID4VC}
  \centering
  \resizebox{\linewidth}{!}{
  
      \begin{sequencediagram}[scale=3]
        \newthread{A}{Usuário}{}
        \newinst{D}{IdP}{}
        \newinst[5]{C}{Carteira}{}
        \newinst[2.75]{B}{Provedor de Serviços}{}
        
        
        \newcounter{counter}
        \stepcounter{counter}
        \begin{call}{A}{(\thecounter) Interação}{B}{Acesso}
            \postlevel
            \stepcounter{counter}
            \begin{call}{B}{(\thecounter) Requisição de Autorização}{C}{\shortstack{(11) Resposta de Autorização \\ (id\_token e  vp\_token com VP)}}
                \postlevel
                \postlevel
    
                \stepcounter{counter}
                \begin{call}{C}{\shortstack{ (\thecounter) Requisição do \\ Objeto de Requisição \\ (contém o DA) }}{B}{\shortstack{(3.5) Resposta com \\ Objeto de Requisição (DA)}}
                \postlevel
                \end{call}
    
                \stepcounter{counter}
                \begin{call}{C}{ (\thecounter) Autenticação e Autorização}{A}{Credenciais e Consentimento}    
                \end{call}
                \postlevel
    
                \stepcounter{counter}
                \begin{call}{A}{\shortstack{(\thecounter) Usuário seleciona credencial}}{C}
                \postlevel
                \end{call}
                \postlevel
                
                \stepcounter{counter}
                \begin{call}{C}{\shortstack{ \shortstack{(\thecounter) Obtém metadados do \\ IdP}}}{D}
                \postlevel
                \end{call}
                \postlevel
    
                \stepcounter{counter}
                \begin{call}{C}{\shortstack{(\thecounter) Requisição do \\ Token de Autorização \\ (Frontend)}}{D}{\shortstack{Token de Autorização}} 
    
                \postlevel
                \stepcounter{counter}
                \begin{call}{D}{ \shortstack{(\thecounter) Autenticação/Autorização}}{A}{\shortstack{Credenciais/Consentimento}}
                \end{call}
                \postlevel
                
                \end{call}
                \postlevel
                \postlevel
    
                \stepcounter{counter}
                \begin{call}{C}{\shortstack{(\thecounter) Requisição do Token de Acesso \\ (Backend)}}{D}{Token de Acesso}
                \end{call}
                \postlevel
                \postlevel
                \postlevel
    
                \stepcounter{counter}
                \begin{call}{C}{\shortstack{(\thecounter) Requisição da emissão de \\ credencial \\ (com o Token de Acesso e a prova)}}{D}{Emissão de credencial}
                \postlevel
                \end{call}
            \end{call}
        \end{call}
      \end{sequencediagram}}
  \label{fig:OID4VC}
  \fonte{O Autor}
\end{figure}
\begin{enumerate}

    \item Usuário acessa um serviço por meio do seu agente (navegador) e usa a Carteira para escanear o QR Code.
    
    \item A Carteira recebe uma Requisição de Autorização. Nela, contém uma URL (Request URL) com os caminhos para pedir o Objeto de Requisição (Request Object).

    \item A Carteira envia uma solicitação GET HTTPS para o URL de solicitação para recuperar o Objeto de Solicitação. A resposta do HTTPS GET retorna o Objeto de Requisição. Esse objeto contém uma \acs{DA}, descrevendo os requisitos das Credenciais que o Provedor de Serviços está solicitando para serem apresentadas. 
    
    Tais requisitos podem incluir o tipo de Credencial, em quais formatos, quais Reivindicações individuais dentro dessas Credenciais (divulgação seletiva). A Carteira processa o Objeto de Requisição e determina quais Credenciais estão disponíveis correspondendo à solicitação do Verificador. 
    
    \item A Carteira autentica o usuário e obtém seu consentimento para apresentar as Credenciais solicitadas.
    
    \item O fluxo iniciado pela Carteira começa quando o utilizador solicita uma Credencial via Carteira ao Provedor de Identidade. O indivíduo seleciona uma credencial de uma lista pré-configurada de credenciais prontas para serem emitidas ou, alternativamente, a carteira o orienta a selecionar uma credencial de um provedor com base nas informações recebidas na solicitação de apresentação de um Provedor de Serviço.


    \item  A Carteira utiliza o URL do Provedor de Identidade para buscar os metadados do mesmo. Os metadados são necessários para aprender os tipos e formatos de Credenciais que o \acs{IdP} suporta. Além disso, a carteira também precisa determinar o endpoint de Autorização (Servidor de Autorização OAuth 2.0) e o endpoint de Credencial necessários para iniciar a solicitação. Observe que nesse exemplo o \acs{IdP} emite credenciais e desempenha o papel de Servidor de Autorização OAuth 2.0, mas poderiam ser entidade diferentes.

    \item \label{item:oidc4vc-item7} A carteira realiza a requisição do Token de Autorização no endpoint de Autorização. O Endpoint de Autorização processa a Solicitação de Autorização. Endpoint utilizado: Token Endpoint.

    \item \label{item:oidc4vc-item8} Para emitir o token de Autorização, é necessário autenticar o usuário e obter seu consentimento para emitir credenciais para a carteira digital.

    \item A Carteira envia uma Solicitação de Token Acesso com o Código de Autorização obtido na etapa anterior. O \acs{IdP} retorna o token de Acesso após validar com sucesso o Código de Autorização.Esta etapa ocorre no backend (comunicação direta entre dois sistemas usando solicitações e respostas HTTP, sem utilizar redirecionamentos através de um intermediário como um navegador). Endpoint utilizado: token endpoint.

    \item A Carteira envia uma Solicitação de emissão de credencial para o \acs{IdP} com o Token de Acesso e (opcionalmente) a prova de posse da chave privada de um par de chaves ao qual ele deve vincular a Credencial emitida. Ao validar com sucesso o Token de Acesso e a prova, o provedor retorna uma credencial.
    
    \item A Carteira prepara as apresentações verificáveis à partir das credenciais verificáveis aos quais o usuário consentiu. Em seguida, envia ao Provedor de Serviços uma Resposta de Autorização onde as apresentações verificáveis estão contidas no parâmetro vp\_token
    
    Observação: As etapas (\ref{item:oidc4vc-item7}) e (\ref{item:oidc4vc-item8}) ocorrem no frontend, redirecionando o usuário por meio do seu navegador.
    
\end{enumerate}
