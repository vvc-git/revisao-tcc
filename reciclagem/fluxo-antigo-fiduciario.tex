\subsubsection{FLUXO DE MENSAGENS}

Abaixo está um diagrama de sequência mostrando as modificações que foram realizadas dentro do OIDC4VP para o Modelo Fiduciário [\ref{fig:OID4VCFiduciary}] . Nele, a Carteira Digital é substítuida pela entidade Fiduciário. 

\begin{figure}

  \centering
  \begin{sequencediagram}[scale=0.1]
    \newthread{A}{Usuário}{}
    \newinst{D}{IdP}{}
    \newinst[5]{C}{Fiduciário}{}
    \newinst[2.75]{B}{Provedor de Serviços}{}
    
    
    \newcounter{number}
    \stepcounter{number}
    \begin{call}{A}{(\thenumber) \color{red} Interação}{B}{Acesso}
        \postlevel
        \stepcounter{number}
        \begin{call}{B}{(\thenumber) Requisição de Autorização}{C}{\shortstack{(9) Resposta de Autorização \\ (id\_token e  vp\_token com VP)}}
            \postlevel
            \postlevel

            \stepcounter{number}
            \begin{call}{C}{\shortstack{ (\thenumber) Requisição do \\ Objeto de Requisição \\ (contém o DA) }}{B}{\shortstack{(3.5) Resposta com \\ Objeto de Requisição (DA)}}
            \postlevel
            \end{call}

            \stepcounter{number}
            \begin{call}{C}{ (\thenumber) Autenticação e Autorização}{A}{Credenciais e Consentimento}    
            \end{call}
            \postlevel

            \stepcounter{number}
            \begin{call}{A}{\shortstack{(\thenumber) Usuário seleciona credencial}}{C}
            \postlevel
            \end{call}
            \postlevel
            
            \stepcounter{number}
            \begin{call}{C}{\shortstack{ \shortstack{(\thenumber) Obtém metadados do \\ IdP}}}{D}
            \postlevel
            \end{call}
            \postlevel

            \stepcounter{number}
            \begin{call}{C}{\shortstack{(\thenumber) \color{blue} Requisição do Token de Acesso \\ \color{blue} (\textbf{com prova de chave})}}{D}{Token de Acesso}
            \end{call}
            \postlevel
            \postlevel
            \postlevel

            \stepcounter{number}
            \begin{call}{C}{\shortstack{(\thenumber) Requisição da emissão de \\ credencial \\ (com o Token de Acesso e a prova)}}{D}{Emissão de credencial}
            \postlevel
            \end{call}
        \end{call}
    \end{call}
  \end{sequencediagram}
  \caption{Diagrama de sequência para o protocolo OID4VC adaptado para o Modelo Fiduciário}
  \label{fig:OID4VCFiduciary}

\end{figure}

\begin{enumerate}

    \item Usuário acessa um serviço por meio do seu agente (geralmente, o \textit{browser}). \textcolor{red}{Essa interação se for feita com o usuário e o fiduciário no mesmo dispositivo DEVE utilizar algo similiar ao \cite{FedCM} e \cite{openYolo}, para que o usuário já encaminhe via Browser o Fiduciário que o Provedor de Serviços irá se comunicar.}   
    \begin{comment}
            \textbf{REVISAR A PARTE 7. WALLET INVOCATION PARA VER SE REALMENTE PRECISA UTILIZAR OUTROS PROTOCOLOS, PORQUE O QUE SE QUER AQUI É CRIAR UM MEIO PARA CHAMAR O FIDUCIÁRIO/CARTEIRA SEM UM REGISTRO PRÉVIO}.
    \end{comment}
    
    \item O Fiduciário recebe uma Requisição de Autorização. Nela, contém uma URL (\textit{Request URL}) com os caminhos para pedir o Objeto de Requisição (\textit{Request Object}). 

    \item o Fiduciário envia uma solicitação GET HTTPS para o URL de solicitação para recuperar o Objeto de Solicitação. A resposta do HTTPS GET retorna o Objeto de Requisição. Esse objeto contém uma Definição de Apresentação (DA), descrevendo os requisitos das Credenciais que o Provedor de Serviços está solicitando para serem apresentadas. 
    
    Tais requisitos podem incluir o tipo de Credencial, em quais formatos, quais Reivindicações individuais dentro dessas Credenciais (divulgação seletiva). o Fiduciário processa o Objeto de Requisição e determina quais Credenciais estão disponíveis correspondendo à solicitação do Verificador. 
    
    \item O Fiduciário autentica o usuário e obtém seu consentimento para apresentar as Credenciais solicitadas.
    
    \item O fluxo iniciado pelo Fiduciário começa quando o Usuário solicita uma Credencial via Fiduciário ao Provedor de Identidade. O usuário seleciona uma credencial de uma lista pré-configurada de Credenciais prontas para serem emitidas, ou alternativamente, o Fiduciário orienta o usuário a selecionar uma credencial de um Provedor de Identidade com base nas informações recebidas na solicitação de apresentação de um Provedor de Serviços.

    \item o Fiduciário utiliza o URL do Provedor de Identidade para buscar os metadados do mesmo. Os metadados são necessários para aprender os tipos e formatos de Credenciais que o Provedor de Identidade suporta. Além disso, o Fiduciário também precisa determinar o \textit{endpoint} de Autorização (Servidor de Autorização OAuth 2.0) e o \textit{endpoint} de Credencial necessários para iniciar a solicitação. Observe que nesse exemplo o Provedor de Identidade emite credenciais e desempenha o papel de Servidor de Autorização OAuth 2.0, mas poderiam ser entidade diferentes.

    \item o Fiduciário envia uma Solicitação de Token Acesso com uma \textcolor{blue}{prova de chave que comprove que o usuário autorizou o Fiduciário a receber uma credencial específica}. O Provedor de identidade retorna o token de Acesso após validar com sucesso essa \textcolor{blue}{prova de chave}. Esta etapa ocorre no \textit{Backend} \textcolor{blue}{e permite eliminar a Requisição de Token de Autorização}, uma vez que o usuário já está autenticado perante o Fiduciário e possui os mecanismos para provar que autorizou emissão de credencias para o seu correspondente. \textit{Endpoint} utilizado: token \textit{endpoint}

    \item o Fiduciário envia uma Solicitação de emissão de credencial para o Provedor de Identidade com o Token de Acesso e (opcionalmente) a prova de posse da chave privada de um par de chaves ao qual o Provedor de Identidade deve vincular a Credencial emitida. Ao validar com sucesso o Token de Acesso e a prova, o Provedor de Identidade retorna uma credencial.
    
    \item o Fiduciário prepara as apresentações verificáveis à partir das credenciais verficicáveis aos quais o usuário consentiu. Em seguida, envia ao Provedor de Serviços uma Resposta de Autorização onde as apresentações verificáveis estão contidas no parâmetro vp\_token
    
    Observação: As etapas (7) e (8) ocorrem no \textit{frontend}, redirecionando o usuário por meio do Agente de Usuário.

\end{enumerate}