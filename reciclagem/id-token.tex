\sigla{OIDC}{Open ID Connect} é um protocolo de autenticação baseado na família de especificações OAuth 2.0. Ele permite que as aplicações autentiquem usuários e obtenham informações sobre eles \cite{openid}, proporcionando uma experiência de \acs{SSO}.

\subsubsection{Papéis}\label{subsubsec:papeis-artefatos-oidc}

Os mesmos papéis presentes no OAuth são mantidos: usuário, cliente, servidor de autorização e servidor de recurso. No entanto, no contexto do OpenID, o servidor de autorização é referido também como Servidor de Autorização.

\subsubsection{Artefatos}\label{subsubsec:artefato-oidc}

\textbf{id\_token}: Token emitido pelo Servidor de Autorização para identificar usuários para a aplicação. Ele é a principal diferença do OAuth e o OIDC: enquanto o primeiro, emite Tokens de Acesso, o segundo, emite id\_Tokens. É representado por \sigla{JWT}{JSON Web token}  que são estruturas JSON divididas em três partes: cabeçalho, corpo e assinatura. 

\begin{table}[ht]
\caption{Parâmetros do id\_token}

\centering
\resizebox{\textwidth}{!}{

    \begin{tabular}{|l|l|}
    
        \hline
        \textbf{Nome} & \textbf{Descrição} \\ \hline
        \textbf{iss} & Autoridade emissora. \\ \hline
        \textbf{sub} & Um identificador único para o usuário emitido pelo emissor. \\ \hline
        \textbf{aud} & Destinatário (aplicação) para a qual este id\_token é destinado. Deve conter o client\_id da aplicação. \\ \hline
        \textbf{exp} & Tempo após o qual o id\_token não deve ser aceito. \\ \hline
        \textbf{iat} & Tempo em que o token é emitido. \\ \hline
        \textbf{auth\_time} & Tempo em que a autenticação do usuário final ocorreu. \\ \hline
    
    \end{tabular}
    
}
\fonte{O Autor}
\label{tab:idtokenparams}
\end{table}
