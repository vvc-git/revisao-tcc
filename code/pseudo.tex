\definecolor{background}{HTML}{EEEEEE}
\colorlet{numb}{magenta!60!black}
\definecolor{fadedtext}{gray}{0.5}

% Definir uma nova linguagem 'pseudo'
\lstdefinelanguage{pseudo}{
  keywords={function, if, else, then, such, that, let, stop, call, otherwise, possible, return}, % Especifica as palavras que devem ser destacadas como palavras-chave.
  sensitive=true,      % Case sensitive
  morecomment=[l]{//}, % Comentários de uma linha
  frame=lines,
}

% Configurações específicas para 'pseudo'
\lstset{
    language=pseudo,
    basicstyle={\tiny\fontfamily{phv}\selectfont},
    keywordstyle=\bfseries\color{black}, % Estilo para palavras-chave
    commentstyle=\color{gray}, % Estilo para comentários
    stringstyle=\color{red}, % Estilo para strings
    mathescape=true, % Permite o uso de notação matemática entre $...$ dentro do ambiente lstlisting.
    backgroundcolor=\color{background},
    numbers=left,                   % Números de linha à esquerda
    numberstyle=\tiny\color{gray},  % Estilo dos números das linhas
    stepnumber=1,                   % Numeração de cada linha (use 2, 3 etc. para numerar a cada 2, 3 linhas)
    literate=
    {:=}{{{\color{numb}{:=}}}}{2}
    % breaklines=true,
    % lineskip=17.99446pt,
}
