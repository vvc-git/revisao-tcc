\usepackage[%
%xindy={language=portuguese}, %para usar o xindy ao invés do makeindex
subentrycounter,
seeautonumberlist,
nonumberlist,
nogroupskip,
shortcuts,   % criação de atalhos
acronym,  % criação de acrônimos  	
translate=babel, % Faz o papel de \providetranslation
nopostdot,  % Não é colocado ponto ao final da entrada
automake
]{glossaries}


\renewcommand{\glossarypreamble}{\vspace{-0.2cm}}  % Retira o espaço proviniente do estilo

% Criação de estilo

\newglossarystyle{mylong1}{% modificado de https://tex.stackexchange.com/a/166209/151962
	\setglossarystyle{long}%
	\renewenvironment{theglossary}%
	{\begin{longtable}[l]{@{}p{\dimexpr 2.5cm-\tabcolsep}p{0.78\hsize}}}% <-- change the value here
		{\end{longtable}}%
}



% Só um pequeno récuo em relação ao primeiro
\newglossarystyle{mylong2}{% modificado de https://tex.stackexchange.com/a/166209/151962
	\setglossarystyle{long}%
	\renewenvironment{theglossary}%
	{\begin{longtable}[l]{@{}p{\dimexpr 2.0cm-\tabcolsep}p{0.86\hsize}}}% <-- change the value here
		{\end{longtable}}%
}



% Criando o comando sigla. Cria a descrição da sigla e a sigla entre parênteses 
% #1 é o atalho da sigla e a sigla. #2 é a descrição 
\newcommand{\sigla}[2]{\newacronym{#1}{#1}{#2}\acrfull{#1}}

% Só dá entrada na lista. Ainda não funciona com o limarka
\newcommand{\siglalista}[2]{\newacronym{#1}{#1}{#2}}  


\newglossary{simbolos}{simbolos}{sbl}{simbolos}

% Criando o comando simbolo. Imprime o símbolo e sua descrição só é mostrada na lista
% #1 é o atalho. #2 é símbolo. #3 descrição 
\newcommand{\simbolo}[3]{\newglossaryentry{#1}{%
						type=simbolos, 
						name=#2, 
						description=#3,
						sort=def}\gls{#1}}

% Só dá entrada na lista. Ainda não funciona com o limarka
\newcommand{\simbololista}[3]{\newglossaryentry{#1}{%
							type=simbolos, 
							name=#2, 
							description=#3,
							sort=def}}

% 
\newcommand{\imprimirlistadesiglas}{%
	    \pdfbookmark[0]{\listadesiglasname}{acn}	
		\printacronyms[%
				style=mylong1,
				title={Lista de Siglas}  % A lista de siglas ainda fica com um espaço a mais
			  ]
		\cleardoublepage
}

\newcommand{\imprimirlistadesimbolos}{%
   	\pdfbookmark[0]{\listadesimbolosname}{sbl}					
	\printglossary[%
			type=simbolos,
			title={\listadesimbolosname }, 
			style=mylong2
			]

	\cleardoublepage
}


\newcommand{\imprimirglossario}{
    \cleardoublepage
   	\pdfbookmark[0]{\glossaryname}{glo}					
	\printglossary[title={\glossaryname}]
}


\makeglossaries
