\newpage
\section{Conclusões} \label{section:conclusion}

O presente trabalho iniciou abordando a complexa tarefa de garantir a identificação digital no cenário atual, ressaltando como a crescente dependência de sistemas digitais nas atividades cotidianas exige um aprofundamento em estudos na área de IAM. Nesse contexto, foram levantados e analisados diferentes modelos de gestão de identidades — centralizado, terceirizado e auto-soberano (SSI) —, culminando na recente proposição do Modelo Fiduciário. Além disso, discutiram-se os principais protocolos que sustentam esses modelos, como OAuth, OpenID Connect (OIDC) e OIDC4VC, destacando suas características e arquitetura. Por fim, foram apresentados os processos e a aplicação de uma análise formal de segurança, ilustrados por meio de um exemplo prático, reforçando a importância de metodologias rigorosas para a validação desses sistemas.

Nesse sentido, o texto apresentou uma adaptação do OpenID Connect com Apresentações Verificáveis para o Modelo Fiduciário, visando fornecer não apenas um mecanismo flexível para a transmissão de VPs, com suporte à divulgação seletiva de informações e a Zero-Knowledge Proofs (ZKP), mas também atender a um dos principais princípios do modelo, o \emph{Non-Disclosure as a Goal}. Para atender a essa demanda, duas extensões foram propostas: \emph{Negociação de Atributos} e \emph{Negociação do Ambiente de Execução}.

A \emph{Negociação de Atributos} é uma solução projetada para reduzir a transmissão de informações desnecessárias do usuário durante o processo de autenticação que utiliza os \texttt{vp\_token}. Para alcançar esse objetivo, a adaptação do OIDC4VP no Modelo Fiduciário foi ajustada, permitindo que o Fiduciário realize requisições ao Provedor de Serviços para modificar o artefato que define as entradas do \texttt{vp\_token} antes de encaminhá-lo para a autenticação do beneficiário. Nesse contexto, foram definidos os formatos das requisições e respostas, bem como os endpoints necessários para viabilizar a implementação dessa funcionalidade de forma eficiente e segura.

A \emph{Negociação do Ambiente de Execução}, por sua vez, é uma solução semelhante à Negociação de Atributos no que diz respeito aos formatos de requisições e respostas, mas com um propósito distinto. Seu objetivo é modificar a lógica referente ao local onde ocorre a manipulação dos dados do usuário, tradicionalmente realizada no serviço web. Essa abordagem permite que o processamento seja executado diretamente no Fiduciário ou de forma colaborativa, utilizando técnicas de Computação Multipartidária para garantir maior controle, privacidade e segurança no tratamento dos dados sensíveis.

Por fim, foi realizada uma análise formal de segurança na extensão proposta para o protocolo OpenID para Apresentações Verificáveis. Para essa análise, definiu-se a propriedade de Autenticação da Negociação e provou-se que ela é mantida dentro dos limites do modelo formal. Essa análise demonstrou que a extensão proposta oferece uma segurança no contexto do modelo definido. Considerando que esses protocolos contribuem significativamente para o avanço do Modelo Fiduciário, esta pesquisa também se torna um incentivo relevante para o desenvolvimento de novas soluções no campo da IAM.

\subsection*{Trabalhos Futuros}

Uma das direções promissoras é investigar o suporte dos protocolos para incorporar regras de consentimento específicas utilizadas pelos Fiduciários. Esse aprimoramento permitiria que os protocolos reconhecessem e aplicassem automaticamente políticas de consentimento conforme definidas por diferentes Fiduciários, promovendo um maior alinhamento com o Modelo Fiduciário.

Outra possibilidade é expandir a análise formal para cobrir propriedades relacionadas à negociação do ambiente de execução. Isso incluiria a definição e validação de propriedades que assegurem a integridade e a conformidade do ambiente em que a negociação ocorre, garantindo que todos os elementos envolvidos na transação sigam as especificações de segurança e privacidade necessárias. Esse tipo de análise pode fortalecer a proteção contra ambientes inseguros ou comprometidos, proporcionando garantias adicionais para as partes envolvidas e consolidando a segurança dos protocolos frente a cenários de execução complexos e distribuídos.