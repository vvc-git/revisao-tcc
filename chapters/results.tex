\section{Resultados e Discussão} \label{section:resultados-discussoes}

Este trabalho apresenta uma adaptação do OpenID Connect com Apresentações Verificáveis dentro do Modelo Fiduciário, resultando em um mecanismo flexível para a transmissão de credenciais e apresentações, com suporte à divulgação seletiva de informações e a \sigla{ZKP}{Prova de Zero-Conhecimento}. As modificações propostas não apenas permitem que os usuários negociem os atributos que desejam disponibilizar ao \acs{SP}, mas também sugerem uma metodologia para a negociação do ambiente de execução conforme descrito pelo novo modelo. Todo o processo foi submetido a uma revisão formal e, subsequentemente, aprimorado para garantir sua robustez e eficácia.